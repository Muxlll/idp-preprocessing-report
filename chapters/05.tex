
\chapter{Conclusion and Outlook}
All in all, in this project a preprocessing algorithm was built as part of a complete machine learning pipeline to predict the hail damage of sugar beets. With this step, the images taken in the mobile application are cropped in a standardized way to improve the damage prediction value. By only training the images with sugar beets in an $ 90° $ angle with the plants centered, the general accuracy of the damage prediction can be improved. Also, with the live predictions in the mobile application, the user can directly interact with the bounding boxes and see whether his camera position and angle is already good or if he still needs to adjust it. \\

By now, this pipeline can only handle sugar beet plants. Of course, this can be further improved by adding other plants if needed. The only requirement is much data of a new plant to add it. For example, a new class of this plant can be added to the detection algorithm to also differentiate the type of plant. Based on this decision, the image could e.g. be sent to a different model trained on the specific plant. \\

Other ideas to improve the models would be to use newer versions of YOLO such as YOLOv6/ YOLOv7. This is an emerging field with very fast development of new feature and more robust models. One such thing is pose detection which could be an interesting new feature. Instead of just predicting bounding boxes, also the concrete structure of for example the leafs would be learned and predicted in new images.