
\chapter{Introduction}

Machine Learning and Computer Vision has gained much importance in the last years. It can be used in many application fields, such as smart plant monitoring. In this context, it can be used to simplify certain workflows. One example is the hail damage detection of sugar beets. 

The idea is to develop a system or application which automatically predicts the damage of plants. This has several advantages, for example less time has to be spent analyzing the fields and possible more accurate results can be achieved. 

More concrete, a mobile application was developed, which allows to take images of sugar beet plants. The fotos taken should then be preprocessed and sent to a backend server which analyzes the images and predicts the damage. 

In the context of this report, the preprocessing step of such a mobile application is presented. The general idea of the preprocessing step in this application is to standardize the image format of the fotos taken. This means that factors like the number of sugar beet plants, the angle in which the foto was taken and many more things are not often optimal for the model that is predicting the damage. In general, the images that we use for training are taken from directly above the plant in a 90 degrees angle. In the best case, the image only contains one plant which is directly in the center. To be therefore consistent with training images, the new pictures of the plants should ideally be in the same format with similar settings. Therefore, the preprocessing step helps to get better results by standardization of the images. 

