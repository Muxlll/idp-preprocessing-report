\chapter{\abstractname}

Machine learning algorithms are gaining more and more importance in many application fields. Also in agriculture, artificial intelligence can be used to improve certain workflows by making them more efficient. 

In this report, a preprocessing algorithm for sugar beet plant detection is presented. It is part of a machine learning pipeline to predict the hail damage of plants in sugar beet fields. The goal of this preprocessing step is to standardize the images taken with the mobile application to improve the results of damage prediction. Available data now has different angles and heights of the camera. By detecting the sugar beets, the images should be modified in a way that only one plant is contained in the center of the image.

The used model is YOLO, a state of the art object detection algorithm. Different aspects of training are compared and results of detecting sugar beet plants are presented. Experiments showed that pretrained models with higher data augmentation achieve the best accuracy. Because of the small difference in predictions of different sized architectures, we decided to focus on a smaller one for live prediction in the mobile application for predicting the damage of sugar beets.